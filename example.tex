\documentclass[letterpaper,openany,nodeprecatedcode]{dndbook}

% Use babel or polyglossia to automatically redefine macros for terms
% Armor Class, Level, etc...
% Default output is in English; captions are located in lib/dndstrings.sty.
% If no captions exist for a language, English will be used.
%1. To load a language with babel:
%	\usepackage[<lang>]{babel}
%2. To load a language with polyglossia:
%	\usepackage{polyglossia}
%	\setdefaultlanguage{<lang>}
\usepackage[english]{babel}
%\usepackage[italian]{babel}
% For further options (multilanguage documents, hypenations, language environments...)
% please refer to babel/polyglossia's documentation.

\usepackage[utf8]{inputenc}
\usepackage[singlelinecheck=false]{caption}
\usepackage{lipsum}
\usepackage{listings}
\usepackage{shortvrb}
\usepackage{stfloats}
\usepackage[percent]{overpic}

\captionsetup[table]{labelformat=empty,font={sf,sc,bf,},skip=0pt}

\MakeShortVerb{|}

\lstset{%
  basicstyle=\ttfamily,
  language=[LaTeX]{TeX},
  breaklines=true,
}

\title{The Dark \LaTeX{} \\
\large An Example of the dndbook Class}
\author{The rpgTeX Team}
\date{2020/04/21}

\begin{document}

\frontmatter

\begin{titlepage}
	\centering
  \begin{overpic}[width=1\textwidth]{img/dragon-door}
    \put (5,80) {
      \vspace{0.5cm}
      {\color{yellow}\fontsize{20}{60}\textsc{An Example of the Dragonbane Class}\par}
    }
    \put (5,75) {
      \vspace{1.5cm}
      {\color{red}\fontsize{40}{60}\bfseries\MakeUppercase{Dragonbane \LaTeX{}}\par}
      \vspace{2cm}
    }
    \put (20,10) {
      {\color{yellow}\fontsize{25}{60}\textsc{Chekmx Publishing} \par}
    }
   \end{overpic}
	
\end{titlepage}

\tableofcontents

\mainmatter%

\part{Layout}

\chapter{\textbf{\MakeUppercase{Sections}}}

\begin{multicols}{2}
  \DndDropCapLine{T}{his package is designed to aid you} in writing beautifully typeset documents for the fifth edition of the world's greatest roleplaying game. It starts by adjusting the section formatting from the defaults in \LaTeX{} to something a bit more familiar to the reader. The chapter formatting is displayed above.
\end{multicols}

\begin{multicols}{2}
  [
      \section{Section}
  ]
  Sections break up chapters into large groups of associated text. In sed pretium erat, consectetur iaculis ligula. Aenean ultrices mauris ut magna porttitor, eget accumsan arcu lacinia. Fusce vel ipsum auctor massa ultrices consectetur sed in odio. Vivamus euismod facilisis ipsum, a volutpat nunc fermentum ac. Praesent suscipit libero urna, vel cursus urna aliquam et. Integer elementum nisi nec placerat volutpat. Ut sed ante a quam scelerisque suscipit. Aenean laoreet ex ac neque pharetra consequat. Cras dictum nec justo et egestas.
  
  \subsection{Subsection}
  Subsections further break down the information for the reader.

  \subsubsection{Subsubsection}
  Subsubsections are the furthest division of text that still have a block header. Below this level, headers are displayed inline.

  \paragraph{Paragraph}
  The paragraph format is seldom used in the core books, but is available if you prefer it to the ``normal'' style.

  \subparagraph{Subparagraph}
  The subparagraph format with the paragraph indent is likely going to be more familiar to the reader.

  \section{Special Sections}
  The module also includes functions to aid in the proper typesetting of multi-line section headers: |\DndFeatHeader| for feats, |\DndItemHeader| magic items and traps, and |\DndSpellHeader| for spells.

  \DndFeatHeader{Typesetting Savant}[Prerequisite: \LaTeX{} distribution]
  You have acquired a package which aids in typesetting source material for one of your favorite games, giving you the following benefits:



  \DoDHeroicAbilityHeader{Typesetting Savant}{ \LaTeX{} 14}{1}
  You have acquired a package which aids in typesetting source material for one of your favorite games, giving you the following benefits:

  \begin{itemize}
    \item You have advantage on Intelligence checks to typeset new content.
    \item When you fail an Intelligence check to typeset new content, you can ask questions online at the package's website.
  \end{itemize}

  \DndItemHeader{Foo's Quill}{Wondrous item, rare}
  This quill has 3 charges. While holding it, you can use an action to expend 1 of its charges. The quill leaps from your hand and writes a contract applicable to your situation.

  The quill regains 1d3 expended charges daily at dawn.

  \DoDSpellHeader%
      {Beautiful Typesetting}
      {1}
      {Any School of Magic}
      {Word, gesture}
      {Action}
      {10 meters}
      {Instant}
  You are able to transform a written message of any length into a beautiful scroll. All creatures within range that can see the scroll must make a wisdom saving throw or be charmed by you until the spell ends.

  While the creature is charmed by you, they cannot take their eyes off the scroll and cannot willingly move away from the scroll. Also, the targets can make a wisdom saving throw at the end of each of their turns. On a success, they are no longer charmed.

  \DoDSkillHeader {Skill} {STAT}
Skill description can be place here when using the |DoDSkillHeader|

\end{multicols}

\chapter{Map Regions}
\begin{multicols}{2}
  \DndDropCapLine{T}{he map region functions} |\DndArea| and |\DndSubArea| provide automatic numbering of areas. \lipsum[4]
\end{multicols}

\begin{multicols}{2}
  \DndArea{Village of Hommlet}
  This is the village of hommlet.  This 

  \DoDSubArea{Inn of the Welcome Wench}
  \DoDItalics{This is the read aloud section for this particular sub location}
  \begin{itemize}
    \DoDBulletGreen \textcolor{DoDGreen} {Normal room:} GM facing item of interest for the given sublocation
    \DoDBulletRed \textcolor{red} {Dangerous room:} Danger! Danger! If you want to replicate the adventure modules formatting of having items that highlight traps or monsters use this
  \end{itemize}

  \DndSubArea{Blacksmith's Forge}
  There's a blacksmith in town, too.

  \DndArea{Foo's Castle}
  This is foo's home, a hovel of mud and sticks.

  \DndSubArea{Moat}
  This ditch has a board spanning it.

  \DndSubArea{Entrance}
  A five-foot hole reveals the dirt floor illuminated by a hole in the roof.
\end{multicols}

\chapter{Text Boxes}

\begin{multicols}{2}
  \DndDropCapLine{T}he module has three environments for setting text apart so that it is drawn to the reader's attention. |DndReadAloud| is used for text that a game master would read aloud.
\end{multicols}

\begin{multicols}{2}
  \begin{DndReadAloud}
    As you approach this module you get a sense that the blood and tears of many generations went into its making. A warm feeling welcomes you as you type your first words.
  \end{DndReadAloud}

  \section{As an Aside}
  The other two environments are the |DndComment| and the |DndSidebar|. The |DndComment| is breakable and can safely be used inline in the text.

  \begin{DndComment}{This Is a Comment Box!}
    A |DndComment| is a box for minimal highlighting of text. It lacks the ornamentation of |DndSidebar|, but it can handle being broken over a column.
  \end{DndComment}

  The |DndSidebar| is not breakable and is best used floated toward a page corner as it is below.

  \begin{DndSidebar}[float=!b]{Behold the DndSidebar!}
    The |DndSidebar| is used as a sidebar. It does not break over columns and is best used with a figure environment to float it to one corner of the page where the surrounding text can then flow around it.
  \end{DndSidebar}

  \section{Tables}
  The |DndTable| colors the even rows and is set to the width of a line by default.

  \begin{DndTable}[header=Nice Table]{XX}
      Table head  & Table head \\
      Some value  & Some value \\
      Some value  & Some value \\
      Some value  & Some value
  \end{DndTable}
\end{multicols}

\chapter{Monsters and NPCs}

\begin{multicols}{2}
  % Monster stat block
  \DndDropCapLine{T}he |DoDMonster| environment is used to typeset monster stat blocks. The module supplies many functions to easily typeset the contents of the stat block
\end{multicols}

% Monster stat block
\begin{DoDMonster}[float*=t,width=\textwidth + 8pt]{Crypt Ooze}
    \begin{multicols}{2}
        \DoDMonsterType{Amorphous transparent blob that engulfs and absorbs carrion. These creature can often be found in catacombs, crypts or graveyards and the remians of there last meal can oft be seen floating in there stomachs}
        % If you want to use commas in the key values, enclose the values in braces.
        \DoDMonsterBasicsTable[
            ferocity = {1},
            size       = {Normal},
            movement = {6},
            hitpoints = {36},
            ]
        \DoDMonsterAction{Blob}
        Amorphous blob, can fit through small gaps, cannot be grappled.
        \DoDMonsterAction{Resistances}
        Takes half damage from Slashing, Piercing, Acid.
        \DoDMonsterAction{Acidic}
        Any one that touches (including attacks) a Crypt Ooze takes 1d4 acid damage.  Any non magic weapon looses one point of durability.
    \end{multicols}
    \begin{DndTable}[header=Monster Attacks, color=PhbTan]{cXX}
        d6 & Attack \\
        \DoDMonsterAttack{1}{Acid Wave}{The Crypt Ooze erupts into a bubbling mass and spits acidic chunks of itself in every direction.  All players within 4m take 1d4 acid damage.  The attack maybe parried by a shield but the shield is corroded and loses the damage from its durability instead.  The blob takes 1 point of damage per player hit.}
        \DoDMonsterAttack{2}{Pseudopod Pounding!}{The Crypt Ooze forms a single massive pseudopod and smashes one of the players.  The victim takes 2d8 bludgeoning damage.}
        \DoDMonsterAttack{3}{Acid Gob!}{The crypt ooze picks a single target with in 10m and gobs a part of itself at the victim.  The victim takes 2d6 acid damage.  Armour protects against this damage but the armour rating is reduced by one.  The attack maybe parried by a shield but the shield is corroded and loses the damage from its durability instead.}
        \DoDMonsterAttack{4}{Slipper when Wet!}{The Ooze secreets a slippery substance from its body onto the floor around it.  Anyone within 4m must roll AGL or fall prone, taking 1d4 bludgeoning damage.}
        \DoDMonsterAttack{5}{Pseudopod Shuffle!}{The crypt forms a number of  pseudopods and slaps any player with 2m.  All victims takes 1d8 bludgeoning damage.}
        \DoDMonsterAttack{6}{Engulf!}{the ooze engulfs one player who takes 2d6 acid damage and is grappled.  Armour protects but the target loses one point of armour rating per round.  Breaking free requires a successful STR roll or an attack that causes slashing damage attack that inflicts 6 points of damage to the Ooze.}
    \end{DndTable}
\end{DoDMonster}  

\begin{multicols}{2}
  % NPC stat block
  The |DoDNPC| environment is used to typeset monster stat blocks. The module supplies many functions to easily typeset the contents of the stat block

  \begin{DoDNPC}{Npc Foo no abilities}
    \DoDNPCType{Lorem ipsum dolor sit amet, consectetur adipiscing elit. Sed a eros faucibus, eleifend magna non, dignissim sem. Aenean fermentum semper elit, a sodales tortor fringilla ut. Quisque accumsan turpis leo, vel sollicitudin nisi bibendum id. Maecenas malesuada, augue at pulvinar maximus, lectus tortor finibus lacus, nec convallis enim est eget lorem. Aliquam eleifend dui eu libero ullamcorper vestibulum. Vivamus metus sapien, luctus id ex id, pharetra rutrum ipsum. Ut gravida viverra justo non molestie.}
    % If you want to use commas in the key values, enclose the values in braces
    \DoDNPCBasicsTable[
        movement = {16},
        dmgBonus = {STR D4},
        armour = {Studded leather and open helmet (3)},
        hitpoints = {16},
        willpoints = 12,
        skills = {Awareness 14, Brawling 14, Evade 12, Healing 8, Hunting \& Fishing 8},
        weapons = {
            \DoDNPCWeapon{Broadsword}{14}{2D6}, 
            \DoDNPCWeapon{Heavy Crossbow}{12}{2D8}, 
            large shield}
    ]
  \end{DoDNPC}


  \begin{DoDNPC}{Npc Foo no abilities}
    \DoDNPCType{Lorem ipsum dolor sit amet, consectetur adipiscing elit. Sed a eros faucibus, eleifend magna non, dignissim sem. Aenean fermentum semper elit, a sodales tortor fringilla ut. Quisque accumsan turpis leo, vel sollicitudin nisi bibendum id. Maecenas malesuada, augue at pulvinar maximus, lectus tortor finibus lacus, nec convallis enim est eget lorem. Aliquam eleifend dui eu libero ullamcorper vestibulum. Vivamus metus sapien, luctus id ex id, pharetra rutrum ipsum. Ut gravida viverra justo non molestie.}
    % If you want to use commas in the key values, enclose the values in braces
    \DoDNPCBasicsTable[
        movement = {16},
        dmgBonus = {STR D4},
        armour = {Studded leather and open helmet (3)},
        hitpoints = {16},
        willpoints = 12,
        skills = {Awareness 14, Brawling 14, Evade 12, Healing 8, Hunting \& Fishing 8},
        weapons = {
            \DoDNPCWeapon{Broadsword}{14}{2D6}, 
            \DoDNPCWeapon{Heavy Crossbow}{12}{2D8}, 
            large shield}
    ]
  \end{DoDNPC}

  Morbi eu tincidunt nisi, quis mollis nisi. Quisque nec dictum nunc. Donec at nisl ipsum. Mauris odio ex, vestibulum et tristique eget, condimentum non neque. Maecenas blandit massa at scelerisque hendrerit. Nunc velit ex, finibus ac dui vitae, tincidunt consequat arcu. Nunc in varius arcu. Aliquam auctor pharetra nisi ac ultrices. Cras blandit leo odio, ac interdum eros iaculis in. Pellentesque lacinia imperdiet orci nec hendrerit. Pellentesque vel tempus elit, id iaculis odio. Sed malesuada vitae turpis ac ullamcorper. In nec egestas nulla. Nullam at justo in eros dictum varius. Proin et ligula sed tellus congue auctor tempus non leo. Etiam tempus nunc in congue sodales.

  \begin{DoDNPC}{Npc Foo no abilities}
    \DoDNPCType{Lorem ipsum dolor sit amet, consectetur adipiscing elit. Sed a eros faucibus, eleifend magna non, dignissim sem. Aenean fermentum semper elit, a sodales tortor fringilla ut. Quisque accumsan turpis leo, vel sollicitudin nisi bibendum id. Maecenas malesuada, augue at pulvinar maximus, lectus tortor finibus lacus, nec convallis enim est eget lorem. Aliquam eleifend dui eu libero ullamcorper vestibulum. Vivamus metus sapien, luctus id ex id, pharetra rutrum ipsum. Ut gravida viverra justo non molestie.}
    % If you want to use commas in the key values, enclose the values in braces
    \DoDNPCBasicsTable[
        movement = {16},
        dmgBonus = {STR D4},
        armour = {Studded leather and open helmet (3)},
        hitpoints = {16},
        willpoints = 12,
        skills = {Awareness 14, Brawling 14, Evade 12, Healing 8, Hunting \& Fishing 8},
        weapons = {
            \DoDNPCWeapon{Broadsword}{14}{2D6}, 
            \DoDNPCWeapon{Heavy Crossbow}{12}{2D8}, 
            large shield}
    ]
  \end{DoDNPC}

  \begin{DoDNPC}[float*=t,width=\textwidth + 8pt]{Npc Foo}
    \begin{multicols}{2}
        \DoDNPCType{Lorem ipsum dolor sit amet, consectetur adipiscing elit. Sed a eros faucibus, eleifend magna non, dignissim sem. Aenean fermentum semper elit, a sodales tortor fringilla ut. Quisque accumsan turpis leo, vel sollicitudin nisi bibendum id. Maecenas malesuada, augue at pulvinar maximus, lectus tortor finibus lacus, nec convallis enim est eget lorem. Aliquam eleifend dui eu libero ullamcorper vestibulum. Vivamus metus sapien, luctus id ex id, pharetra rutrum ipsum. Ut gravida viverra justo non molestie.}
        % If you want to use commas in the key values, enclose the values in braces.
    \end{multicols}
    \DoDNPCBasicsWideTable[
        movement = {16},
        dmgBonus = {STR D4},
        armour = {Studded leather and open helmet (3)},
        hitpoints = {16},
        willpoints = 12,
        skills = {Awareness 14, Brawling 14, Evade 12, Healing 8},
        abilities = {Defensive, Veteran},
        weapons = {
            \DoDNPCWeapon{Broadsword}{14}{2D6}, 
            \DoDNPCWeapon{Heavy Crossbow}{12}{2D8}, 
            large shield}
    ]
  \end{DoDNPC}  

  \pagebreak
  \subsubsection*{Black Powder Weapons}
  All weapons use exploding dice for damage and have the following features,  Piercing, Load, 

  \rowcolors {1} {} {PhbTan}
  \begin{table*}[b]
    \caption{\DndFontTableTitle{}Black Powder Weapons}\label{tab:colors}
    \begin{tabularx}{\linewidth}{llllXcXXXX}
      \textbf{Weapon} & \textbf{Grip} & \textbf{STR} & \textbf{Range} & \textbf{Damage} & \textbf{Durability} & \textbf{Cost} & \textbf{Supply} & \textbf{Features} & \textbf{Melee}\\
      Pistol & 1H & 7 & 20 & 3d4& 6 & 300 & Rare & Load & Small Club \\
      Carbine & 2H & 7 & 30 & 4d4& 7 & 500 & Rare & Load & Staff Spear \\
      Musket & 3H & 7 & 40 & 4d4& 9 & 500 & Rare & Load & Staff Spear \\
      Dragon & 1H & 7 & 15 & 3/2/1d6& 6 & 500 & Rare & Load, shot & Small Club \\
      Donerbuss & 2H & 7 & 20 & 4/3/2/1d6& 7 & 300 & Rare & Load, shot & Large Club \\
      Grenado & 1H & 9 & STR & 5d4& - & 50 & Rare & Explosive & - \\
    \end{tabularx}
  \end{table*}
\end{multicols}

\part{Customization}

\chapter{Colors}
\begin{multicols}{2}
  [
    \begin{multicols}{2}
      This package provides several global color variables to style |DndComment|, |DndReadAloud|, |DndSidebar|, and |DndTable| environments.
    \end{multicols}
  ]
  \begin{table*}[b]
    \caption{\DndFontTableTitle{}Colors Supported by this Package}\label{tab:colors}

    \begin{tabularx}{\linewidth}{lX}
      \textbf{Color}                  & \textbf{Description} \\
      \rowcolor{PhbLightGreen}
      |PhbLightGreen|                 & Light green used in PHB Part 1 (Default) \\
      \rowcolor{PhbLightCyan}
      |PhbLightCyan|                  & Light cyan used in PHB Part 2 \\
      \rowcolor{PhbMauve}
      |PhbMauve|                      & Pale purple used in PHB Part 3 \\
      \rowcolor{PhbTan}
      |PhbTan|                        & Light brown used in PHB appendix \\
      \rowcolor{DmgLavender}
      |DmgLavender|                   & Pale purple used in DMG Part 1 \\
      \rowcolor{DmgCoral}
      |DmgCoral|                      & Orange-pink used in DMG Part 2 \\
      \rowcolor{DmgSlateGray}
      |DmgSlateGray| (|DmgSlateGrey|) & Blue-gray used in PHB Part 3 \\
      \rowcolor{DmgLilac}
      |DmgLilac|                      & Purple-gray used in DMG appendix \\
      \rowcolor{BrGreen}
      |BrGreen|                       & Gray-green used for tables in Basic Rules\\
    \end{tabularx}
  \end{table*}
  
  \begin{DndTable}[header=Box Colors]{lX}
    Color            &  Description \\
    |commentcolor|   & |DndComment| background \\
    |readaloudcolor| & |DndReadAloud| background \\
    |sidebarcolor|   & |DndSidebar| background \\
    |tablecolor|     & background of even |DndTable| rows \\
  \end{DndTable}

  They also accept an optional color argument to set the color for a single instance. See Table~\ref{tab:colors} for a list of core book accent colors.

  \begin{lstlisting}
  \begin{DndTable}[color=PhbLightCyan]{cX}
    d8 & Item \\
    1  & Small wooden button \\
    2  & Red feather \\
    3  & Human tooth \\
    4  & Vial of green liquid\\
    5  & Loaded dice \\
    6  & Tasty biscuit \\
    7  & Broken axe handle \\
    8  & Tarnished silver locket \\
  \end{DndTable}
  \end{lstlisting}

  \begin{DndTable}[color=PhbLightCyan]{cX}
    d8 & Item \\
    1  & Small wooden button\\
    2  & Red feather \\
    3  & Human tooth \\
    4  & Vial of green liquid \\
    5  & Loaded dice \\
    6  & Tasty biscuit \\
    7  & Broken axe handle \\
    8  & Tarnished silver locket \\
  \end{DndTable}


  \section{Themed Colors}
  Use |\DndSetThemeColor[<color>]| to set |commentcolor|, |readaloudcolor|, |sidebarcolor|, and |tablecolor| to a specific color. Calling |\DndSetThemeColor| without an argument sets those colors to the current |themecolor|. In the following example the group limits the change to just a few boxes; after the group finishes, the colors are reverted to what they were before the group started.

  \begin{lstlisting}
    \begingroup
      \DndSetThemeColor[PhbMauve]

      \begin{DndComment}{This Comment Is in Mauve}
        This comment is in the the new color.
      \end{DndComment}

      \begin{DndSidebar}{This Sidebar Is Also Mauve}
        The sidebar is also using the new theme color.
      \end{DndSidebar}
    \endgroup
  \end{lstlisting}

  \begingroup
    \DndSetThemeColor[PhbMauve]

    \begin{DndComment}{This Comment Is in Mauve}
      This comment is in the the new color.
    \end{DndComment}

    \begin{DndSidebar}{This Sidebar Is Also Mauve}
      The sidebar is also using the new theme color.
    \end{DndSidebar}
  \endgroup

  \end{multicols}
\end{document}
